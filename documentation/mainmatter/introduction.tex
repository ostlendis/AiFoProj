\chapter{Introduction}
\label{chapter:introduction}

This document details the implementation process for the AI Foundations module Miniproject. The project involves generating and evaluating at least three ideas that leverage a GPT-based assistant and building an application around the chosen concept.

The process begins with idea generation and selection based on defined evaluation criteria. Once an idea is selected, we refine it and develop specifications for the application.

Our goal was to choose an idea with practical utility—something we would personally use. During brainstorming, we considered common challenges where a large language model (LLM) could assist us. Within a short time, we generated three ideas:

\begin{itemize}
    \item A food recipe generator that provides recipes based on ingredients available at home (user input).
    \item A gift idea generator, inspired by the upcoming holiday season, which suggests gifts based on characteristics of the recipient.
    \item A belt balancer generator for the game \href{https://www.factorio.com}{Factorio}\footnote{Not affiliated with the product; however, we recommend trying it.}.
\end{itemize}

Next, we evaluated the practicality and feasibility of each idea. While all were potentially useful, the belt balancer idea presented significant complexity and implementation challenges. A belt balancer in Factorio helps distribute items across multiple input and output belts. Factorio allows players to import such structures as blueprint strings, which are encoded in base64 and compressed using zlib before translating into JSON for in-game structures.

Although it initially seemed promising, generating a valid belt balancer string with an LLM proved difficult. Even after ten attempts, none of the generated strings imported successfully into the game, so we decided to eliminate this idea.

We established the following criteria to evaluate the remaining ideas:

\begin{itemize}
    \item Practical use (weight: 30)
    \item Ease of implementation (weight: 5)
    \item Benefit (weight: 20)
    \item Originality (weight: 10)
\end{itemize}

\begin{table}[!h]
    \begin{tabular}{|lllll|r|}
        \hline
        Idea                & Practical Use   & Ease of Implementation & Benefit         & Originality     & Total Score  \\
        \hline
        Recipe Generator    & 5 * \textbf{30} & 8 * \textbf{5}         & 7 * \textbf{20} & 5 * \textbf{10} & \textbf{400} \\
        \hline
        Gift Idea Generator & 7 * \textbf{30} & 8 * \textbf{5}         & 9 * \textbf{20} & 8 * \textbf{10} & \textbf{510} \\
        \hline
    \end{tabular}
\end{table}

The Gift Idea Generator scored highest, so we selected it as our project.

Before implementation, we defined specifications for the application. We agreed on the following requirements:
The application should accept three input parameters: Gender, Age, and Personal Interests. It should also be accessible on mobile devices for convenient use on the go. When the user enters the parameters and presses the button to generate ideas, they should receive five gift suggestions in text form.
